\begin{frame}{Quantization of Continuous-Time Control}

	\vspace{0em}

	\begin{block}{Quantization Problem}
		{
		Given a continuous-time optimal control signal
		$u : x \to \mathbb{R}^m$,
		we approximate it by a \emph{piecewise-constant} control
		$\overline u$ with $N$ updates,
		by minimizing an $L^p$ approximation error:
		{
		\medskip

		% \resizebox{\textwidth}{!}{
		\(
		\left\{
		\begin{array}{lcll}
			\displaystyle
			\min_{\{\tau_k,\,u_k\}}
			\;
			\int_{\Omega}
			\underbrace{
				\| u(x) - \overline u(x) \|_p
			}_{\substack{\text{Approximation} \\ \text{error}}}
			\, dx
			\\
			\text{s.t. }\quad
			\overline u(x) = u_k,\quad
			x \in [\tau_k, \tau_{k+1}),
			 & k = 0,\ldots,N-1
		\end{array}
		\right.
		\)
		% }
		}
		}
	\end{block}

	\vspace{0.5em}

	\only<1>{
		\textbf{Idea:}
		replace the continuous optimal solution by a piecewise-constant one
		that is \emph{closest in the $L^p(\Omega)$ sense}.
	}

\end{frame}

\questionslide{
	\begin{itemize}
		\item is $\min\limits_{\{\tau_k,\,u_k\}}$ valid? or should be $\{\tau_1, \ldots, \tau_k\}$ instead?
		\item $\Omega$ undefined is fine?
	\end{itemize}
}

% ============================================================================================

\begin{frame}{Quantization Problem: Example}

	\begin{example}[Input space dimension $m=1$]
		\centering
		\begin{tikzpicture}[
				scale=0.9,
				every node/.style={scale=1.05},
				>=Stealth
			]

			\begin{scope}[xshift=-4.5cm]
				\centering

				\node (cont) at (2,0) {
					$
						\min \limits_{u}
						\displaystyle\int_{\Omega}
						\big| u(x) - \overline u(x) \big|^2
						\, dx
					$
				};


				\node[scale=\ArrowScale] at (2,-1.0) {$\Downarrow$};

				\node (sum) at (2,-2.2) {
					$
						\displaystyle\sum_{k=0}^{N-1}
						\displaystyle\int_{\tau_k}^{\tau_{k+1}}
						\big| u(x) - u_k \big|^2
						\, dx
					$
				};

				\node[gray] (partition) at (6.4,-2.2) {partition $x$};
				\draw[->, gray] (partition) -- (sum);

				\node[scale=\ArrowScale] at (2,-3.3) {$\Downarrow$};

				\node[red] (opt) at (5.8,-3.2) {\emph{Optimality condition}};
				\draw[->, red] (opt) -- (2.5,-3.2);
				\node (cond) at (2,-5.2) {
					\begin{equationbox}[7cm]
						\left| u_{k-1} - u(t_k) \right|^2
						=
						\left| u_k - u(t_k) \right|^2
					\end{equationbox}
				};

			\end{scope}
		\end{tikzpicture}

	\end{example}

	\vspace{0.4em}

\end{frame}

% =======================================================================================

\begin{frame}{Quantization: Intuition}
	\centering

	\begin{equationbox}[7cm]
		\left| u_{k-1} - u(t_k) \right|^2
		=
		\left| u_k - u(t_k) \right|^2
	\end{equationbox}
	\centering
	\includegraphics[width=0.9\textwidth]{resources/quant.png}
\end{frame}



% =====================================================================================

\begin{frame}{Quantization-Based Sampling: Algorithm ($m=1$, $p=2$)}


	\begin{block}{Iterative Procedure}
		\only<1>{\footnotesize}\only<2->{\tiny}
		\begin{enumerate}
			\item \textbf{Initialization}
			      Set
			      \[
				      t_k^{(0)} = \frac{k}{N}T,
				      \quad k=0,\ldots,N .
			      \]

			\item \textbf{Centroid update} (fix $\{t_k^{(r)}\}$)
			      For each interval $k=0,\ldots,N-1$, compute
			      \[
				      \overline u_k^{(r+1)}
				      =
				      \frac{1}{t_{k+1}^{(r)} - t_k^{(r)}}
				      \int_{t_k^{(r)}}^{t_{k+1}^{(r)}}
				      u(t)\, dt .
			      \]

			\item \textbf{Boundary update} (fix $\{u_k^{(r+1)}\}$)
			      For each $k=1,\ldots,N-1$, update $t_k^{(r+1)}$ as the solution of
			      \[
				      u(t_k)
				      =
				      \frac{\overline u_{k-1}^{(r+1)} + \overline u_k^{(r+1)}}{2},
				      \qquad
				      t_{k-1}^{(r+1)} < t_k < t_{k+1}^{(r)} .
			      \]

			\item \textbf{Repeat} steps 2–3 until convergence.
		\end{enumerate}
	\end{block}


	\vspace{0.3em}

	\uncover<2>{
		\begin{columns}
			\begin{column}{0.7\textwidth}

				{
					\color{gray!80}
					\textit{
						$\ldots$ centroids minimize error inside intervals, boundaries balance errors between neighboring intervals.
					}
				}
			\end{column}
		\end{columns}
	}



\end{frame}

\questionslide{
	\begin{itemize}
		\item does the algorithm actually make sense? it was not in the paper explicitly, I think.
		\item Is the algorithm too small on the second slide?
		\item We cannot use a screenshot and need a proper plot
	\end{itemize}

}


%
