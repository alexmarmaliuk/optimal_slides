\begin{frame}{Optimal Sampling}

\begin{block}{Discrete-Time Optimal Cost}
For a fixed sampling pattern $\{\tau_0,\dots,\tau_{N-1}\}$,
the optimal cost is
\[
\mathcal{J} = x_0^\top \overline{K}_0 x_0,
\]
where $\overline{K}_k$ satisfies the Riccati recursion
\[
\overline{K}_k = r(\tau_k, \overline{K}_{k+1}), 
\overline{K}_N = S.
\]

\end{block}


$\overline{K}_k$ depends only on the current and future
sampling intervals $\{\tau_k,\tau_{k+1},\dots,\tau_{N-1}\}$.
Hence,
\[
h < k \;\Rightarrow\;
\frac{\partial \overline{K}_k}{\partial \tau_h} = 0.
\]



\end{frame}

%%%%%%%%%%%%%%%%%%%%%%%%%%%%%%%%%%%%%%%%%%%%%%%%%%%%%%%%%%%%%%%%%%%%%%%%%%%%%%%%%%%%%%%%%%%%%%%%%%%%%%%%%%%%%%%%%%%%%%%%%%%%%%%%%%%%%%%%%%%%%%%%%%%%%%%%%%%%%%%%%%%%%%%%%%%%%%%%%%%%%%%%%%%%%%%%%%%%%%%%%%%%%%%


\begin{frame}{Optimal Sampling: KKT Optimality Condition}


    The sampling intervals satisfy
\[
\sum_{k=0}^{N-1} \tau_k = T.
\]
At the optimal sampling pattern, the gradient of the cost
must be proportional to $[1,\,1,\,\dots,\,1]$
which implies that
\[
\frac{\partial \mathcal{J}}{\partial \tau_h}
=
\frac{\partial \mathcal{J}}{\partial \tau_{h+1}},
\qquad h = 0,\dots,N-2.
\]
Since $\mathcal{J} = x_0^\top \bar{K}_0 x_0$, the above condition is equivalent to
\[
x_0^\top
\frac{\partial \bar{K}_0}{\partial \tau_h}
=
x_0^\top
\frac{\partial \bar{K}_0}{\partial \tau_{h+1}},
\]
which, after expanding the Riccati recursion, leads to the optimality condition

\end{frame}

%%%%%%%%%%%%%%%%%%%%%%%%%%%%%%%%%%%%%%%%%%%%%%%%%%%%%%%%%%%%%%%%%%%%%%%%%%%%%%%%%%%%%%%%%%%%%%%%%%%%%%%%%%%%%%%%%%%%%%%%%%%%%%%%%%%%%%%%%%%%%%%%%%%%%%%%%%%%%%%%%%%%%%%%%%%%%%%%%%%%%%%%%%%%%%%%%%%%%%%%%%%%%%%



%%%%%%%%%%%%%%%%%%%%%%%%%%%%%%%%%%%%%%%%%%%%%%%%%%%%%%%%%%%%%%%%%%%%%%%%%%%%%%%%%%%%%%%%%%%%%%%%%%%%%%%%%%%%%%%%%%%%%%%%%%%%%%%%%%%%%%%%%%%%%%%%%%%%%%%%%%%%%%%%%%%%%%%%%%%%%%%%%%%%%%%%%%%%%%%%%%%%%%%%%%%%%%%


%%%%%%%%%%%%%%%%%%%%%%%%%%%%%%%%%%%%%%%%%%%%%%%%%%%%%%%%%%%%%%%%%%%%%%%%%%%%%%%%%%%%%%%%%%%%%%%%%%%%%%%%%%%%%%%%%%%%%%%%%%%%%%%%%%%%%%%%%%%%%%%%%%%%%%%%%%%%%%%%%%%%%%%%%%%%%%%%%%%%%%%%%%%%%%%%%%%%%%%%%%%%%%%



\begin{frame}
\begin{block}{Riccati-Based Condition}
Using the chain rule on the Riccati recursion, this condition
can be rewritten as
\[
			x_0' \prod_{i=0}^{h-1} \left[ \frac{\partial r}{\partial \overline{K}} (\tau_i, \overline{K}_{i+1}) \right]
			\left[
				\frac{\partial r}{\partial \tau}
				\bigl(\tau_h,\, \overline{K}_{h+1}\bigr)
				-
				\frac{\partial r}{\partial \overline{K}}
				\bigl(\tau_h,\, \overline{K}_{h+1}\bigr)
				\,
				\frac{\partial r}{\partial \tau}
				\bigl(\tau_{h+1},\, \overline{K}_{h+2}\bigr)
				\right]
			= 0
		\]

\end{block}

\end{frame}
%%%%%%%%%%%%%%%%%%%%%%%%%%%%%%%%%%%%%%%%%%%%%%%%%%%%%%%%%%%%%%%%%%%%%%%%%%%%%%%%%%%%%%%%%%%%%%%%%%%%%%%%%%%%%%%%%%%%%%%%%%%%%%%%%%%%%%%%%%%%%%%%%%%%%%%%%%%%%%%%%%%%%%%%%%%%%%%%%%%%%%%%%%%%%%%%%%%%%%%%%%%%%%%

\begin{frame}{Optimal Sampling: First Order Systems}
    \begin{block}{Discrete-Time Riccati Recursion}
        In discrete time, the Riccati recursion has the form:
        \[
            \overline{K}_k = r(\tau_k, \overline{K}_{k+1})
        \]
        
        \vspace{-1em}
        For a first-order system ($n=1$), the recurrence function $r$ becomes 
        a scalar rational function of $\tau$ and $\overline{K}$.

        \[r(\tau, \overline{K}) = \frac{ \overline{Q}_k \overline{R}_k - \overline{P}_k^2 + (\overline{A}_k^2 \overline{R}_k - 2 \overline{A}_k \overline{B}_k \overline{P}_k + \overline{B}_k^2 \overline{Q}_k) \overline{K}
        }{ \overline{R}_k + \overline{B}_k^2 \overline{K} }.
        \]

        \vspace{-1em}
        \bigskip
        With partial derivatives of $r(\tau, \overline{K}): \frac{\partial r}{\partial \tau}, \frac{\partial r}{\partial \overline{K}}$

    \end{block}
\end{frame}

%%%%%%%%%%%%%%%%%%%%%%%%%%%%%%%%%%%%%%%%%%%%%%%%%%%%%%%%%%%%%%%%%%%%%%%%%%%%%%%%%%%%%%%%%%%%%%%%%%%%%%%%%%%%%%%%%%%%%%%%%%%%%%%%%%%%%%%%%%%%%%%%%%%%%%%%%%%%%%%%%%%%%%%%%%%%%%%%%%%%%%%%%%%%%%%%%%%%%%%%%%%%%%% Normalized Cost %%%%%%%%%%%%%%%%%%%%%%%%%%%%%%%%%%%%%%%%%%%%%%%%%%%%%%%%%%%%%%%%%%%%%%%%%%%%%%%%%%%%%%%%%%%%%%%%%%%%%%%%%%%%%%%%%%%%%%%%%%%%%%%%%%%%%%%%%%%%%%%%%%%%%%%%%%%%%%%%%%%%%%%%%%%%%%%%%%%%%%%%%%%%%%%%%%%%%%%%%%%%%%%

\begin{frame}{Necessary Condition for Optimal Sampling}
\begin{block}{General Optimality Condition}
The general necessary condition for optimality is
\[
\frac{\partial \mathcal{J}}{\partial \tau_h} = 0.
\]
\end{block}

Using the Riccati recursion and the chain rule, this condition becomes
\[
\frac{\partial r}{\partial \tau}
\bigl(\tau_h,\, r(\tau_{h+1}, \overline{K}_{h+2})\bigr)
-
\frac{\partial r}{\partial \overline{K}}
\bigl(\tau_h,\, r(\tau_{h+1}, \overline{K}_{h+2})\bigr)
\cdot
\frac{\partial r}{\partial \tau}
\bigl(\tau_{h+1},\, \overline{K}_{h+2}\bigr)
= 0
\]

\end{frame}

%%%%%%%%%%%%%%%%%%%%%%%%%%%%%%%%%%%%%%%%%%%%%%%%%%%%%%%%%%%%%%%%%%%%%%%%%%%%%%%%%%%%%%%%%%%%%%%%%%%%%%%%%%%%%%%%%%%%%%%%%%%%%%%%%%%%%%%%%%%%%%%%%%%%%%%%%%%%%%%%%%%%%%%%%%%%%%%%%%%%%%%%%%%%%%%%%%%%%%%%%%%%%%% Normalized Cost %%%%%%%%%%%%%%%%%%%%%%%%%%%%%%%%%%%%%%%%%%%%%%%%%%%%%%%%%%%%%%%%%%%%%%%%%%%%%%%%%%%%%%%%%%%%%%%%%%%%%%%%%%%%%%%%%%%%%%%%%%%%%%%%%%%%%%%%%%%%%%%%%%%%%%%%%%%%%%%%%%%%%%%%%%%%%%%%%%%%%%%%%%%%%%%%%%%%%%%%%%%%%%%

\begin{frame}{Quantization-Based Sampling}
\begin{block}{Quantization based Sampling}
Quantization-based sampling approximates a continuous input $u(t)$
by a piecewise-constant function $\overline{u}(t)$ with $N$ values,
aiming to minimize the approximation error. This method provides a near-minimal cost.
\end{block}

The quantization error is defined as
\[
E_{\text{qnt}} =
\sum_{k=0}^{N-1}
\int_{t_k}^{t_{k+1}}
|u(t) - u_k|^2 \, dt.
\]

\begin{itemize}
\item Unknowns: constants $u_k$ and instants $t_k$
\item Both values and sampling instants are optimized
\end{itemize}
\end{frame}

%%%%%%%%%%%%%%%%%%%%%%%%%%%%%%%%%%%%%%%%%%%%%%%%%%%%%%%%%%%%%%%%%%%%%%%%%%%%%%%%%%%%%%%%%%%%%%%%%%%%%%%%%%%%%%%%%%%%%%%%%%%%%%%%%%%%%%%%%%%%%%%%%%%%%%%%%%%%%%%%%%%%%%%%%%%%%%%%%%%%%%%%%%%%%%%%%%%%%%%%%%%%%%% Normalized Cost %%%%%%%%%%%%%%%%%%%%%%%%%%%%%%%%%%%%%%%%%%%%%%%%%%%%%%%%%%%%%%%%%%%%%%%%%%%%%%%%%%%%%%%%%%%%%%%%%%%%%%%%%%%%%%%%%%%%%%%%%%%%%%%%%%%%%%%%%%%%%%%%%%%%%%%%%%%%%%%%%%%%%%%%%%%%%%%%%%%%%%%%%%%%%%%%%%%%%%%%%%%%%%%

\begin{frame}{Quantization: Optimality Conditions (m=1)}
Differentiating $E_{\text{qnt}}$ with respect to $u_k$ yields
\[
u_k =
\frac{1}{t_{k+1}-t_k}
\int_{t_k}^{t_{k+1}} u(t)\,dt.
\]

\bigskip

Differentiating $E_{\text{qnt}}$ with respect to $t_k$ gives
\[
|u_{k-1} - u(t_k)|^2
=
|u_k - u(t_k)|^2
\]

\begin{itemize}
\item This method is appplicable for any linear system with dimension of input space $m = 1$ and any dimension $n$ of the state space.
\item Sampling instants are chosen so that the approximation error is
balanced across intervals.
\end{itemize}

\end{frame}

%%%%%%%%%%%%%%%%%%%%%%%%%%%%%%%%%%%%%%%%%%%%%%%%%%%%%%%%%%%%%%%%%%%%%%%%%%%%%%%%%%%%%%%%%%%%%%%%%%%%%%%%%%%%%%%%%%%%%%%%%%%%%%%%%%%%%%%%%%%%%%%%%%%%%%%%%%%%%%%%%%%%%%%%%%%%%%%%%%%%%%%%%%%%%%%%%%%%%%%%%%%%%%%
\begin{frame}{Quantization-Based Sampling: Optimality Condition}

\begin{align*}
u_k
&=
\frac{1}{t_{k+1}-t_k}
\int_{t_k}^{t_{k+1}} u(t)\,dt
\\[1.2ex]
E_{\text{qnt}}
&=
\int_0^T |u(t)|^2 \, dt
-
\sum_{k=0}^{N-1}
(t_{k+1}-t_k)\,|u_k|^2
\\[1.2ex]
&=
-|u_k|^2
+
2u'(t_k)u_k
+
|u_{k-1}|^2
-
2u'(t_k)u_{k-1}
\\[1.2ex]
&=
|u_{k-1}-u(t_k)|^2
-
|u_k-u(t_k)|^2
\\[1.2ex]
&\Longrightarrow
|u_{k-1}-u(t_k)|^2
-
|u_k-u(t_k)|^2
\end{align*}

\end{frame}

%%%%%%%%%%%%%%%%%%%%%%%%%%%%%%%%%%%%%%%%%%%%%%%%%%%%%%%%%%%%%%%%%%%%%%%%%%%%%%%%%%%%%%%%%%%%%%%%%%%%%%%%%%%%%%%%%%%%%%%%%%%%%%%%%%%%%%%%%%%%%%%%%%%%%%%%%%%%%%%%%%%%%%%%%%%%%%%%%%%%%%%%%%%%%%%%%%%%%%%%%%%%%%%



\begin{frame}{Asymptotic Quantization}
As $N \to \infty$, $\tau_k = 0$:
\begin{itemize}
\item Sampling intervals shrink
\item Piecewise-constant approximation becomes dense
\end{itemize}

For a scalar input ($m=1$), known results from optimal quantization imply
\[
\sigma(t) \propto |\dot{u}(t)|^{\frac{p}{p+1}}.
\]

Choosing $p=2$, we obtain the quantization density
\[
\sigma_{\text{qnt}}(t) =
\frac{
|\dot{u}(t)|^{2/3}
}{
\int_0^T |\dot{u}(s)|^{2/3} ds
}
\]

\end{frame}
%%%%%%%%%%%%%%%%%%%%%%%%%%%%%%%%%%%%%%%%%%%%%%%%%%%%%%%%%%%%%%%%%%%%%%%%%%%%%%%%%%%%%%%%%%%%%%%%%%%%%%%%%%%%%%%%%%%%%%%%%%%%%%%%%%%%%%%%%%%%%%%%%%%%%%%%%%%%%%%%%%%%%%%%%%%%%%%%%%%%%%%%%%%%%%%%%%%%%%%%%%%%%%%

\begin{frame}{Asymptotic Quantization}
\small
\[
\frac{f(y)^{\frac{m}{m+p}}}
{\int f(y)^{\frac{m}{m+p}}\,dy}
\]

\[
m = 1
\qquad\Longrightarrow\qquad
f(y) = \frac{1}{|\dot u|}\,u^{-1}(y)
\]

\[
\frac{|\dot u|^{-\frac{1}{1+p}}\,u^{-1}(y)}
{\int |\dot u|^{-\frac{1}{1+p}}\,u^{-1}(y)\,dy}
\]

\[
\sigma_{\mathrm{qnt}}(t)
=
\frac{|\dot u(t)|^{\frac{p}{1+p}}}
{\int_0^T |\dot u(t)|^{\frac{p}{1+p}}\,dt}
\]

\[
p = 2
\qquad\Longrightarrow\qquad
\sigma_{\mathrm{qnt}}(t)
=
\frac{|\dot u(t)|^{\frac{2}{3}}}
{\int_0^T |\dot u(s)|^{\frac{2}{3}}\,ds}
\]

\[
\int_{t_k}^{t_{k+1}} |\dot u(t)|^{\frac{2}{3}}\,dt
=
\frac{1}{N}
\int_0^T |\dot u(t)|^{\frac{2}{3}}\,dt
\]

\end{frame}







%%%%%%%%%%%%%%%%%%%%%%%%%%%%%%%%%%%%%%%%%%%%%%%%%%%%%%%%%%%%%%%%%%%%%%%%%%%%%%%%%%%%%%%%%%%%%%%%%%%%%%%%%%%%%%%%%%%%%%%%%%%%%%%%%%%%%%%%%%%%%%%%%%%%%%%%%%%%%%%%%%%%%%%%%%%%%%%%%%%%%%%%%%%%%%%%%%%%%%%%%%%%%%%

\begin{frame}{Lemma 6}

    For the first-order systems, the asymptotic quantization density $\sigma_{\text{qnt}}$ coincides with the optimal sampling density $\sigma_{\text{opt}}$.

\bigskip
{Assumptions (for simplification):}
\begin{itemize}
\item $B = 1$, $R = 1$
\item $Q = 0$, $A > 0$
\item $S = K_\infty = A + \sqrt{A^2 + Q}$
\end{itemize}
\bigskip
These assumptions enables us to have an expression for optimal continuous-time input u:
$u(t) = -x_0 (A + \sqrt{A^2 + Q}) e^{-\sqrt{A^2 + Q}\, t}.$
\end{frame}

%%%%%%%%%%%%%%%%%%%%%%%%%%%%%%%%%%%%%%%%%%%%%%%%%%%%%%%%%%%%%%%%%%%%%%%%%%%%%%%%%%%%%%%%%%%%%%%%%%%%%%%%%%%%%%%%%%%%%%%%%%%%%%%%%%%%%%%%%%%%%%%%%%%%%%%%%%%%%%%%%%%%%%%%%%%%%%%%%%%%%%%%%%%%%%%%%%%%%%%%%%%%%%%

\begin{frame}{Asymptotic Optimal Sampling Density}
For small sampling intervals:
\begin{itemize}
\item $A_k R_k \neq B_k P_k$
\item $\tau_h \to \tau_{h+1}$ as both tend to zero
\end{itemize}

We obtain the expansion
\[
\tau_h
=
\alpha \tau_{h+1}
+
\beta \tau_{h+1}^2
+
o(\tau_{h+1}^2)
\]


Equating constant terms yields
\[
\alpha = 1,
\qquad
\beta = -\frac{2}{3}
\frac{(Q+A^2)\overline{K}_{h+2}}{Q + A \overline{K}_{h+2}}.
\]
\end{frame}

%%%%%%%%%%%%%%%%%%%%%%%%%%%%%%%%%%%%%%%%%%%%%%%%%%%%%%%%%%%%%%%%%%%%%%%%%%%%%%%%%%%%%%%%%%%%%%%%%%%%%%%%%%%%%%%%%%%%%%%%%%%%%%%%%%%%%%%%%%%%%%%%%%%%%%%%%%%%%%%%%%%%%%%%%%%%%%%%%%%%%%%%%%%%%%%%%%%%%%%%%%%%%%%

\begin{frame}{Asymptotic Optimal Sampling Density}
\small
 
The asymptotic sampling density of the optimal pattern
$\sigma_{\text{opt}}(t)$ satisfies the differential equation
\[
\dot{\sigma}_{\text{opt}}(t)
=
-\frac{2}{3}
\frac{(Q+A^2)\,K(t)}{Q + A K(t)}
\,\sigma_{\text{opt}}(t),
\]
where $K(t)$ is the solution of the continuous-time Riccati equation.


When $S = K_\infty = A + \sqrt{A^2 + Q}$
\[
\dot{\sigma}_{\text{opt}}(t)
=
-\frac{2}{3}\sqrt{Q+A^2}\,\sigma_{\text{opt}}(t)
\]

\[
\frac{\dot{\sigma}_{\text{opt}}(t)}{\sigma_{\text{opt}}(t)}
=
-\frac{2}{3}\sqrt{Q+A^2}
\]

\[
\int \frac{\dot{\sigma}_{\text{opt}}(t)}{\sigma_{\text{opt}}(t)}\,dt
=
-\frac{2}{3}\sqrt{Q+A^2}\,t
\]

\[
\ln \sigma_{\text{opt}}(t)
=
-\frac{2}{3}\sqrt{Q+A^2}\,t + \ln c
\]

\[
\sigma_{\text{opt}}(t)
=
c\,e^{-\frac{2}{3}\sqrt{Q+A^2}\,t}
\]

\[
1
=
\int_0^T \sigma_{\text{opt}}(t)\,dt
=
c \int_0^T e^{-\frac{2}{3}\sqrt{Q+A^2}\,t}\,dt
\]

\[
=
c
\left[
-\frac{3}{2\sqrt{Q+A^2}}
e^{-\frac{2}{3}\sqrt{Q+A^2}\,t}
\right]_0^T
\]

\[
=
c
\frac{3}{2\sqrt{Q+A^2}}
\left(
1 - e^{-\frac{2}{3}\sqrt{Q+A^2}\,T}
\right)
\]

\[
\Rightarrow\quad
c
=
\frac{2\sqrt{Q+A^2}}{3}
\frac{1}{1 - e^{-\frac{2}{3}\sqrt{Q+A^2}\,T}}.
\]



\begin{block}{Key Result}
Using the expression of the optimal continuous-time input,
\[
\sigma_{\text{opt}}(t) \;\propto\; |\dot{u}(t)|^{2/3}.
\]

\end{block}

\end{frame}




%%%%%%%%%%%%%%%%%%%%%%%%%%%%%%%%%%%%%%%%%%%%%%%%%%%%%%%%%%%%%%%%%%%%%%%%%%%%%%%%%%%%%%%%%%%%%%%%%%%%%%%%%%%%%%%%%%%%%%%%%%%%%%%%%%%%%%%%%%%%%%%%%%%%%%%%%%%%%%%%%%%%%%%%%%%%%%%%%%%%%%%%%%%%%%%%%%%%%%%%%%%%%%%


\begin{frame}{Lemma 7: Asymptotic Normalized Cost}
\begin{block}{Lemma 7}
For a sampling method $m_\alpha$ with asymptotic density
\[
\sigma_{m_\alpha}(t)
=
\frac{\alpha(K_\infty-A)}{1 - e^{-\alpha(S-A)T}}
e^{-\alpha(K_\infty-A)t}
\propto |\dot{u}(t)|^\alpha,
\]
the asymptotic normalized cost is
\[
c_{m_\alpha}
=
\frac{S}{12 (K_\infty-A) T^2}
\,
\frac{1 - e^{-2(1-\alpha)(K_\infty-A)T}}{2(1-\alpha)}
\,
\frac{1 - e^{-\alpha (K_\infty-A)T}}{\alpha}
\]

\end{block}
\end{frame}





%%%%%%%%%%%%%%%%%%%%%%%%%%%%%%%%%%%%%%%%%%%%%%%%%%%%%%%%%%%%%%%%%%%%%%%%%%%%%%%%%%%%%%%%%%%%%%%%%%%%%%%%%%%%%%%%%%%%%%%%%%%%%%%%%%%%%%%%%%%%%%%%%%%%%%%%%%%%%%%%%%%%%%%%%%%%%%%%%%%%%%%%%%%%%%%%%%%%%%%%%%%%%%%


\begin{frame}{Results}

\begin{itemize}
    \item Optimal sampling for linear control systems depends on the system dynamics and cost.

    \item For first-order systems, the optimal sampling pattern can be
    characterized by an \textbf{asymptotic sampling density}.

    \item The optimal density admits the closed-form structure
    \[
        \sigma_{\text{opt}}(t) \propto |\dot{u}(t)|^{2/3}
    \]

    \item This density coincides with that obtained from
    \textbf{quantization-based sampling} of the optimal continuous-time input.

    \item Optimal and quantization-based sampling achieve a
    \textbf{lower asymptotic cost} than periodic sampling.
\end{itemize}


\end{frame}

