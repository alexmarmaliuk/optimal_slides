\begin{frame}{Optimal Sampling}
	
	\begin{block}{Discrete-Time Optimal Cost}
		For a fixed sampling pattern $\{\tau_0,\dots,\tau_{N-1}\}$,
		the optimal cost is
		\[
			\mathcal{J} = x_0^\top \bar{K}_0 x_0,
		\]
		where $\bar{K}_k$ satisfies the Riccati recursion
		\[
			\bar{K}_k = r(\tau_k, \bar{K}_{k+1}), 
			\bar{K}_N = S.
		\]
		
	\end{block}
	
	
	$\bar{K}_k$ depends only on the current and future
	sampling intervals $\{\tau_k,\tau_{k+1},\dots,\tau_{N-1}\}$.
	Hence,
	\[
		h < k \;\Rightarrow\;
		\frac{\partial \bar{K}_k}{\partial \tau_h} = 0.
	\]
	
	
	
\end{frame}

%%%%%%%%%%%%%%%%%%%%%%%%%%%%%%%%%%%%%%%%%%%%%%%%%%%%%%%%%%%%%%%%%%%%%%%%%%%%%%%%%%%%%%%%%%%%%%%%%%%%%%%%%%%%%%%%%%%%%%%%%%%%%%%%%%%%%%%%%%%%%%%%%%%%%%%%%%%%%%%%%%%%%%%%%%%%%%%%%%%%%%%%%%%%%%%%%%%%%%%%%%%%%%%


\begin{frame}{Optimal Sampling: Necessary Optimality Condition}
	
	\begin{block}{KKT Optimality Condition}
		The sampling intervals satisfy
		\[
			\sum_{k=0}^{N-1} \tau_k = T.
		\]
		At the optimal sampling pattern, the gradient of the cost
		must be proportional to $[1,\,1,\,\dots,\,1]$
		which implies that
		\[
			\frac{\partial \mathcal{J}}{\partial \tau_h}
			=
			\frac{\partial \mathcal{J}}{\partial \tau_{h+1}},
			\qquad h = 0,\dots,N-2.
		\]
	\end{block}
\end{frame}

%%%%%%%%%%%%%%%%%%%%%%%%%%%%%%%%%%%%%%%%%%%%%%%%%%%%%%%%%%%%%%%%%%%%%%%%%%%%%%%%%%%%%%%%%%%%%%%%%%%%%%%%%%%%%%%%%%%%%%%%%%%%%%%%%%%%%%%%%%%%%%%%%%%%%%%%%%%%%%%%%%%%%%%%%%%%%%%%%%%%%%%%%%%%%%%%%%%%%%%%%%%%%%%

\begin{frame}
	\begin{block}{Riccati-Based Condition}
		Using the chain rule on the Riccati recursion, this condition
		can be rewritten as
		\[
			x_0^\top
			\left[
				\frac{\partial r}{\partial \tau}
				\bigl(\tau_h,\, \bar{K}_{h+1}\bigr)
				-
				\frac{\partial r}{\partial \bar{K}}
				\bigl(\tau_h,\, \bar{K}_{h+1}\bigr)
				\,
				\frac{\partial r}{\partial \tau}
				\bigl(\tau_{h+1},\, \bar{K}_{h+2}\bigr)
				\right]
			= 0
		\]
		
	\end{block}
	
\end{frame}
%%%%%%%%%%%%%%%%%%%%%%%%%%%%%%%%%%%%%%%%%%%%%%%%%%%%%%%%%%%%%%%%%%%%%%%%%%%%%%%%%%%%%%%%%%%%%%%%%%%%%%%%%%%%%%%%%%%%%%%%%%%%%%%%%%%%%%%%%%%%%%%%%%%%%%%%%%%%%%%%%%%%%%%%%%%%%%%%%%%%%%%%%%%%%%%%%%%%%%%%%%%%%%%

\begin{frame}{Optimal Sampling: First Order Systems}
	\begin{block}{Discrete-Time Riccati Recursion}
		In discrete time, the Riccati recursion has the form:
		\[
			\bar{K}_k = r(\tau_k, \bar{K}_{k+1})
		\]
		
		\vspace{-1em}
		For a first-order system ($n=1$), the recurrence function $r$ becomes 
		a scalar rational function of $\tau$ and $\bar{K}$.
		
		\[r(\tau, \bar{K}) = \frac{ \bar{Q}_k \bar{R}_k - \bar{P}_k^2 + (\bar{A}_k^2 \bar{R}_k - 2 \bar{A}_k \bar{B}_k \bar{P}_k + \bar{B}_k^2 \bar{Q}_k) \bar{K}
			}{ \bar{R}_k + \bar{B}_k^2 \bar{K} }.
		\]
		
		\vspace{-1em}
		\bigskip
		With partial derivatives of $r(\tau, \bar{K}): \frac{\partial r}{\partial \tau}, \frac{\partial r}{\partial \bar{K}}$
		
	\end{block}
\end{frame}

%%%%%%%%%%%%%%%%%%%%%%%%%%%%%%%%%%%%%%%%%%%%%%%%%%%%%%%%%%%%%%%%%%%%%%%%%%%%%%%%%%%%%%%%%%%%%%%%%%%%%%%%%%%%%%%%%%%%%%%%%%%%%%%%%%%%%%%%%%%%%%%%%%%%%%%%%%%%%%%%%%%%%%%%%%%%%%%%%%%%%%%%%%%%%%%%%%%%%%%%%%%%%%% Normalized Cost %%%%%%%%%%%%%%%%%%%%%%%%%%%%%%%%%%%%%%%%%%%%%%%%%%%%%%%%%%%%%%%%%%%%%%%%%%%%%%%%%%%%%%%%%%%%%%%%%%%%%%%%%%%%%%%%%%%%%%%%%%%%%%%%%%%%%%%%%%%%%%%%%%%%%%%%%%%%%%%%%%%%%%%%%%%%%%%%%%%%%%%%%%%%%%%%%%%%%%%%%%%%%%%

\begin{frame}{Necessary Condition for Optimal Sampling}
	\begin{block}{General Optimality Condition}
		The general necessary condition for optimality is
		\[
			\frac{\partial \mathcal{J}}{\partial \tau_h} = 0.
		\]
	\end{block}
	
	Using the Riccati recursion and the chain rule, this condition becomes
	\[
		\frac{\partial r}{\partial \tau}
		\bigl(\tau_h,\, r(\tau_{h+1}, \bar{K}_{h+2})\bigr)
		-
		\frac{\partial r}{\partial \bar{K}}
		\bigl(\tau_h,\, r(\tau_{h+1}, \bar{K}_{h+2})\bigr)
		\cdot
		\frac{\partial r}{\partial \tau}
		\bigl(\tau_{h+1},\, \bar{K}_{h+2}\bigr)
		= 0
	\]
	
\end{frame}

%%%%%%%%%%%%%%%%%%%%%%%%%%%%%%%%%%%%%%%%%%%%%%%%%%%%%%%%%%%%%%%%%%%%%%%%%%%%%%%%%%%%%%%%%%%%%%%%%%%%%%%%%%%%%%%%%%%%%%%%%%%%%%%%%%%%%%%%%%%%%%%%%%%%%%%%%%%%%%%%%%%%%%%%%%%%%%%%%%%%%%%%%%%%%%%%%%%%%%%%%%%%%%% Normalized Cost %%%%%%%%%%%%%%%%%%%%%%%%%%%%%%%%%%%%%%%%%%%%%%%%%%%%%%%%%%%%%%%%%%%%%%%%%%%%%%%%%%%%%%%%%%%%%%%%%%%%%%%%%%%%%%%%%%%%%%%%%%%%%%%%%%%%%%%%%%%%%%%%%%%%%%%%%%%%%%%%%%%%%%%%%%%%%%%%%%%%%%%%%%%%%%%%%%%%%%%%%%%%%%%

\begin{frame}{Quantization-Based Sampling}
	\begin{block}{Quantization based Sampling}
		Quantization-based sampling approximates a continuous input $u(t)$
		by a piecewise-constant function $\bar{u}(t)$ with $N$ values,
		aiming to minimize the approximation error. This method provides a near-minimal cost.
	\end{block}
	
	The quantization error is defined as
	\[
		E_{\text{qnt}} =
		\sum_{k=0}^{N-1}
		\int_{t_k}^{t_{k+1}}
		|u(t) - u_k|^2 \, dt.
	\]
	
	\begin{itemize}
		\item Unknowns: constants $u_k$ and instants $t_k$
		\item Both values and sampling instants are optimized
	\end{itemize}
\end{frame}

%%%%%%%%%%%%%%%%%%%%%%%%%%%%%%%%%%%%%%%%%%%%%%%%%%%%%%%%%%%%%%%%%%%%%%%%%%%%%%%%%%%%%%%%%%%%%%%%%%%%%%%%%%%%%%%%%%%%%%%%%%%%%%%%%%%%%%%%%%%%%%%%%%%%%%%%%%%%%%%%%%%%%%%%%%%%%%%%%%%%%%%%%%%%%%%%%%%%%%%%%%%%%%% Normalized Cost %%%%%%%%%%%%%%%%%%%%%%%%%%%%%%%%%%%%%%%%%%%%%%%%%%%%%%%%%%%%%%%%%%%%%%%%%%%%%%%%%%%%%%%%%%%%%%%%%%%%%%%%%%%%%%%%%%%%%%%%%%%%%%%%%%%%%%%%%%%%%%%%%%%%%%%%%%%%%%%%%%%%%%%%%%%%%%%%%%%%%%%%%%%%%%%%%%%%%%%%%%%%%%%

\begin{frame}{Quantization: Optimality Conditions}
	Differentiating $E_{\text{qnt}}$ with respect to $u_k$ yields
	\[
		u_k =
		\frac{1}{t_{k+1}-t_k}
		\int_{t_k}^{t_{k+1}} u(t)\,dt.
	\]
	
	\bigskip
	
	Differentiating $E_{\text{qnt}}$ with respect to $t_k$ gives
	\[
		|u_{k-1} - u(t_k)|^2
		=
		|u_k - u(t_k)|^2
	\]
	
	\begin{itemize}
		\item This method is appplicable for any linear system with dimension of input space $m = 1$ and any dimension $n$ of the state space.
		\item Sampling instants are chosen so that the approximation error is
		      balanced across intervals.
	\end{itemize}
	
\end{frame}

%%%%%%%%%%%%%%%%%%%%%%%%%%%%%%%%%%%%%%%%%%%%%%%%%%%%%%%%%%%%%%%%%%%%%%%%%%%%%%%%%%%%%%%%%%%%%%%%%%%%%%%%%%%%%%%%%%%%%%%%%%%%%%%%%%%%%%%%%%%%%%%%%%%%%%%%%%%%%%%%%%%%%%%%%%%%%%%%%%%%%%%%%%%%%%%%%%%%%%%%%%%%%%%

\begin{frame}{Asymptotic Quantization}
	As $N \to \infty$, $\tau_k = 0$:
	\begin{itemize}
		\item Sampling intervals shrink
		\item Piecewise-constant approximation becomes dense
		\item Sampling pattern converges to a sampling density
	\end{itemize}
	
	For a scalar input ($m=1$), known results from optimal quantization imply
	\[
		\sigma(t) \propto |\dot{u}(t)|^{\frac{p}{p+1}}.
	\]
	
	Choosing $p=2$ (norm = $2$), we obtain the quantization density
	\[
		\sigma_{\text{qnt}}(t) =
		\frac{
			|\dot{u}(t)|^{2/3}
		}{
			\int_0^T |\dot{u}(s)|^{2/3} ds
		}
	\]
	
\end{frame}


% Question slide
\begin{frame}{NOTES}
	\begin{questionbox}
		Explanation for denominator s and should eqn.28 be added?(sampling instants)
	\end{questionbox}
\end{frame}



%%%%%%%%%%%%%%%%%%%%%%%%%%%%%%%%%%%%%%%%%%%%%%%%%%%%%%%%%%%%%%%%%%%%%%%%%%%%%%%%%%%%%%%%%%%%%%%%%%%%%%%%%%%%%%%%%%%%%%%%%%%%%%%%%%%%%%%%%%%%%%%%%%%%%%%%%%%%%%%%%%%%%%%%%%%%%%%%%%%%%%%%%%%%%%%%%%%%%%%%%%%%%%%

\begin{frame}{Lemma 6}
	
	For the first-order systems, the asymptotic quantization density $\sigma_{\text{qnt}}$ coincides with the optimal sampling density $\sigma_{\text{opt}}$.
	
	\bigskip
	{Assumptions (for simplification):}
	\begin{itemize}
		\item $B = 1$, $R = 1$
		\item $Q = 0$, $A > 0$
		\item $S = K_\infty = A + \sqrt{A^2 + Q}$
	\end{itemize}
	\bigskip
	These assumptions enables us to have an expression for optimal continuous-time input u:
	$u(t) = -x_0 (A + \sqrt{A^2 + Q}) e^{-\sqrt{A^2 + Q}\, t}.$
\end{frame}

%%%%%%%%%%%%%%%%%%%%%%%%%%%%%%%%%%%%%%%%%%%%%%%%%%%%%%%%%%%%%%%%%%%%%%%%%%%%%%%%%%%%%%%%%%%%%%%%%%%%%%%%%%%%%%%%%%%%%%%%%%%%%%%%%%%%%%%%%%%%%%%%%%%%%%%%%%%%%%%%%%%%%%%%%%%%%%%%%%%%%%%%%%%%%%%%%%%%%%%%%%%%%%%

\begin{frame}{Asymptotic Optimal Sampling Density}
	For small sampling intervals:
	\begin{itemize}
		\item $A_k R_k \neq B_k P_k$
		\item $\tau_h \to \tau_{h+1}$ as both tend to zero
	\end{itemize}
	
	We obtain the expansion
	\[
		\tau_h
		=
		\alpha \tau_{h+1}
		+
		\beta \tau_{h+1}^2
		+
		o(\tau_{h+1}^2)
	\]
	
	
	Equating constant terms yields
	\[
		\alpha = 1,
		\qquad
		\beta = -\frac{2}{3}
		\frac{(Q+A^2)\bar{K}_{h+2}}{Q + A \bar{K}_{h+2}}.
	\]
\end{frame}

%%%%%%%%%%%%%%%%%%%%%%%%%%%%%%%%%%%%%%%%%%%%%%%%%%%%%%%%%%%%%%%%%%%%%%%%%%%%%%%%%%%%%%%%%%%%%%%%%%%%%%%%%%%%%%%%%%%%%%%%%%%%%%%%%%%%%%%%%%%%%%%%%%%%%%%%%%%%%%%%%%%%%%%%%%%%%%%%%%%%%%%%%%%%%%%%%%%%%%%%%%%%%%%

\begin{frame}{Asymptotic Optimal Sampling Density}
	
	
	The asymptotic sampling density of the optimal pattern
	$\sigma_{\text{opt}}(t)$ satisfies the differential equation
	\[
		\dot{\sigma}_{\text{opt}}(t)
		=
		-\frac{2}{3}
		\frac{(Q+A^2)\,K(t)}{Q + A K(t)}
		\,\sigma_{\text{opt}}(t),
	\]
	where $K(t)$ is the solution of the continuous-time Riccati equation.
	
	
	When $S = K_\infty = A + \sqrt{A^2 + Q}$
	\[
		\dot{\sigma}_{\text{opt}}(t)
		=
		-\frac{2}{3}\sqrt{Q+A^2}\,\sigma_{\text{opt}}(t),
	\]
	results in
	\[
		\sigma_{\text{opt}}(t)
		=
		c\,e^{-\frac{2}{3}\sqrt{Q+A^2}\,t},
		\qquad
		\int_0^T \sigma_{\text{opt}}(t)\,dt = 1.
	\]
	
	
	\begin{block}{Key Result}
		Using the expression of the optimal continuous-time input,
		\[
			\sigma_{\text{opt}}(t) \;\propto\; |\dot{u}(t)|^{2/3}.
		\]
		
	\end{block}
	
\end{frame}


% Question slide
\begin{frame}{NOTES}
	\begin{questionbox}
		Layout issue in the Key result block
	\end{questionbox}
\end{frame}

%%%%%%%%%%%%%%%%%%%%%%%%%%%%%%%%%%%%%%%%%%%%%%%%%%%%%%%%%%%%%%%%%%%%%%%%%%%%%%%%%%%%%%%%%%%%%%%%%%%%%%%%%%%%%%%%%%%%%%%%%%%%%%%%%%%%%%%%%%%%%%%%%%%%%%%%%%%%%%%%%%%%%%%%%%%%%%%%%%%%%%%%%%%%%%%%%%%%%%%%%%%%%%%


\begin{frame}{Lemma 7: Asymptotic Normalized Cost}
	\begin{block}{Lemma 7}
		For a sampling method $m_\alpha$ with asymptotic density
		\[
			\sigma_{m_\alpha}(t)
			=
			\frac{\alpha(S-A)}{1 - e^{-\alpha(S-A)T}}
			e^{-\alpha(S-A)t}
			\propto |\dot{u}(t)|^\alpha,
		\]
		the asymptotic normalized cost is
		\[
			c_{m_\alpha}
			=
			\frac{S}{12 (S-A) T^2}
			\,
			\frac{1 - e^{-2(1-\alpha)(S-A)T}}{2(1-\alpha)}
			\,
			\frac{1 - e^{-\alpha (S-A)T}}{\alpha}
		\]
		
	\end{block}
\end{frame}



% Question slide
\begin{frame}{NOTES}
	\begin{questionbox}
		Are backup slides enough for lemma 6 and lemma 7?
	\end{questionbox}
\end{frame}

%%%%%%%%%%%%%%%%%%%%%%%%%%%%%%%%%%%%%%%%%%%%%%%%%%%%%%%%%%%%%%%%%%%%%%%%%%%%%%%%%%%%%%%%%%%%%%%%%%%%%%%%%%%%%%%%%%%%%%%%%%%%%%%%%%%%%%%%%%%%%%%%%%%%%%%%%%%%%%%%%%%%%%%%%%%%%%%%%%%%%%%%%%%%%%%%%%%%%%%%%%%%%%%


\begin{frame}{Results}
	
	\begin{itemize}
		\item Optimal sampling for linear control systems depends on the system dynamics and cost.
		      
		\item For first-order systems, the optimal sampling pattern can be
		      characterized by an \textbf{asymptotic sampling density}.
		      
		\item The optimal density admits the closed-form structure
		      \[
			      \sigma_{\text{opt}}(t) \propto |\dot{u}(t)|^{2/3}.
		      \]
		      
		\item This density coincides with that obtained from
		      \textbf{quantization-based sampling} of the optimal continuous-time input.
		      
		\item Optimal and quantization-based sampling achieve a
		      \textbf{lower asymptotic cost} than periodic sampling.
	\end{itemize}
	
	
\end{frame}

% Question slide
\begin{frame}{NOTES}
	\begin{questionbox}
		Is result content fine or should I add more explanation?
	\end{questionbox}
\end{frame}