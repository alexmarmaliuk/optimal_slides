% Mathematical 


\newcommand{\dif}{\mathop{}\!\mathrm{d}} % Differential - d
\newcommand{\dgr}{^{\circ}} % Degree sign
\newcommand{\degre}{\ensuremath{^\circ}} % Another degree sign (for GeoGebra exports)
\newcommand{\mvect}[1]{\left(\begin{matrix} #1 \end{matrix}\right)} % Vector in matrix notation
\newcommand{\abs}[1]{\left| #1 \right|} % Setzt den Inhalt in Betragsstriche
\newcommand{\brk}[1]{\left( #1 \right)} % Setzt den Inhalt in große Klammern
\newcommand{\set}[1]{\left\lbrace #1 \right\rbrace} % Setzt den Inhalt in Mengenklammern (geschweifte Klammern)
\newcommand{\where}{\;\middle|\;} % Mengenbedingung { | } (innerhalb von \set{})

\newcommand{\curly}[1]{\left\lbrace#1\right\rbrace} % Geschweifte Klammern
\newcommand{\angles}[1]{\left\langle#1\right\rangle} % Winkelklammern
\newcommand{\floor}[1]{\left\lfloor#1\right\rfloor} % Gaußklammern unten
\newcommand{\ceil}[1]{\left\lceil#1\right\rceil} % Gaußklammern oben
\newcommand{\sema}[1]{\llbracket #1\rrbracket} % Semantikklammern
\newcommand{\semantics}[2][]{\left\llbracket#2\ifthenelse{\isempty{#1}}{\right\rrbracket}{\right\rrbracket_{#1}}}

\newcommand{\NN}{\mathbb{N}} % Mengenzeichen: Natürliche Zahlen
\newcommand{\ZZ}{\mathbb{Z}} % Mengenzeichen: Ganze Zahlen
\newcommand{\QQ}{\mathbb{Q}} % Mengenzeichen: Rationale Zahlen
\newcommand{\RR}{\mathbb{R}} % Mengenzeichen: Reelle Zahlen
\newcommand{\CC}{\mathbb{C}} % Mengenzeichen: Komplexe Zahlen
\newcommand{\PP}{\mathbb{P}} % Mengenzeichen: Primzahlen
\newcommand{\HH}{\mathbb{H}} % Mengenzeichen: Quarternionen
\newcommand{\BB}{\mathbb{B}} % Mengenzeichen: Booleans
\newcommand{\universe}{\mathcal{U}} % Universe
\newcommand{\powerset}[1]{\mathcal{P}\left(#1\right)} % Power set
\newcommand{\rel}{\mathcal{R}}
\newcommand{\dom}{\mathrm{dom}} % Domain of def. (Definitionsbereich)
\newcommand{\im}{\mathrm{im}} % Image (Wertebereich)
\newcommand{\Id}{\mathrm{Id}} % Identity relation
\newcommand{\unirel}{\mathrm{U}} % Univeral relation
\newcommand{\refrel}{\mathrm{Ref}} % Reflexive relation

\newcommand{\true}{\mathtt{true}} % True
\newcommand{\false}{\mathtt{false}} % False

\newcommand{\lnor}{\mathbin{\overline{\lor}}}
\newcommand{\lnand}{\mathbin{\overline{\land}}}

\newcommand{\tx}[1]{\times 10^{#1}} % Powers of ten (scientific notation)

\newcommand{\lists}{\mathcal{L}} % List
\newcommand{\trees}{\mathcal{T}} % Tree
\newcommand{\binarytrees}{\mathcal{B}} % Binary tree
\newcommand{\listsof}[1]{\mathcal{L}\round{#1}} % List of


% Rechenschritte in align
\newcommand{\resetaligncount}{\setcounter{equation}{0}} % Setzt den Gleichungszähler zurück
\newcommand{\ad}[1]{&& | \: #1} % Beschreibung für einen Schritt (Mathe)
\newcommand{\adt}[1]{&& | \: \text{#1}} % Beschreibung für einen Schritt (Text)

% Statistik
\newcommand{\EE}{\mathbb{E}} % Erwartungswert
\newcommand{\Cov}{\mathrm{Cov}} % Covarianz
\newcommand{\Ave}{\mathrm{Ave}} % Mittelwert
\newcommand{\Var}{\mathrm{Var}} % Varianz

% Prog1-Definitions
\newcommand{\nil}{[]} %{\textsf{nil}}
\newcommand{\eset}{\ensuremath{\emptyset}}
\newcommand{\cc}{\,{:}\,}
\newcommand{\ltd}{{\mathrel{\dot{<}}}}
\newcommand{\cld}{,\dots,}  % ,...,
\DeclareMathOperator\eva{\,\triangleright\,}

\newcommand{\mi}[1]{\mathit{#1}}
\newcommand{\bnfmid}{\:\mid\:}